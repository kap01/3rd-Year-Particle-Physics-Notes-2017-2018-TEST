\section{Natural units}
\label{sec:naturalUnits}
We will usually use ``natural'' units with 
\[
c=\hbar=1.
\]
 Disclaimer: If inconsistencies are found in the notes in terms of missing/additional
 factors of $\hbar$ or $c$ please let me know. The person that spots the most errors
 will be in for a prize (nothing too special, more about the 
 feeling of satisfaction and pride...) 
 
 Amongst other things, this implies that time and length have the same
 units. From
\[
  E^2 = (cp)^2 + (mc^2)^2,
\]
 which simplifies to
\[
  E^2 = p^2 + m^2
\]
 in our system of units, we see that energy, momentum and mass
 have the same units. The relationship between wavelength and
 momentum of the photon:
\[
   p = \hbar \frac{2\pi}{\lambda}
\]
 implies that the units for momentum are inverse to those for
 distance. If we choose \units{GeV} as our energy unit, we get:
\begin{itemize}
  \item \units{GeV} as unit for mass, energy, momentum
  \item \units{GeV^{-1}} as unit for time and distance
\end{itemize}

The trick for translating from natural units, where $\hbar=c=1$, to
other systems, is to multiply any part of the expression/formula in
natural units with factors of $1$ expressed in terms of $\hbar$, $c$,
$c^2$, $\hbar c$, etc, until the expression has the correct units in
both systems. Remember that the units of $\hbar$ are [Energy]$\times$[Time] (e.g. \units{J\, s} or \units{MeV \, s} in SI units) while the units of $c$ are [Distance]/[Time]. Their product has units of [Energy]$\times$[Distance]. A relationship that is particularly useful for translating those units back to SI
 units is
\[
  \hbar c = \un{196}{MeV\,fm}.
\]
It's worth remembering this, or at least
\[
  \hbar c \approx \un{200}{MeV\,fm}.
\]

Let's for example translate the lifetime of a particle
from \units{MeV^{-1}} into seconds. The width of the \prt{\phi} particle is
$\Gamma = \un{4.458}{MeV}$, so, in our system of units, its lifetime
is $\tau = 1/\Gamma=\un{0.2243}{MeV^{-1}} =\un{0.2243}{MeV^{-1}} \hbar
= \un{0.2243}{MeV^{-1}} \hbar c/c$. The expression on the right is
exactly the same as the one on the left in our system of units, but it
also has the correct units in the SI system (time). Hence
 \begin{eqnarray*}
\tau &=&       \un{0.22}{MeV^{-1}} \hbar c/c                 \\
     &\approx& \un{0.22}{MeV^{-1}} \cdot \un{200}{MeV\,fm} /c      \\
     &\approx&       \un{44}{fm} /c                                \\
     &\approx& \un{44/3 \cdot 10^{-15}\cdot 10^{-8}}{s}  \\
     &\approx&        1.5 \cdot 10^{-22} \units{s}
\end{eqnarray*}

\exercise{
Translate the following recipe into natural units:
"Take 2 eggs, 500g of flour and a pint of milk. Mix for 10 min."
\vspace{1ex}\\
\rotatebox{180}{\parbox{0.9\textwidth}{
\begin{itemize}
\item 2 is just a number and remains invariant under the unit change.
\item 500$g$ is a mass, we want to know what that is in energy units, so we need to calculate: $mc^2 = \un{0.5}{kg}\cdot \un{9\cdot 10^{16}}{m^2/s^2} = \un{4.5\cdot 10^{16}}{J}$. Now multiply by $e/e$: $\un{4.5\cdot 10^{16}}{e J/e}$. Use $J/C = V$ and $e=\un{1.6 \cdot 10^{-19}}{C}$ to get
\un{2.8\cdot 10^{35}}{eV} of flour.
\item 1pint refers to a volume of \un{568}{cm^3}. We will measure volume in terms of $(eV)^-3$. We'll use that $\hbar c$ has units of [energy]$\times$[length]. So...
$\un{568\cdot 10^{-6}}{m^3} / (\hbar c)^3 = 
\un{5.68\cdot 10^{-4}}{m^3} / (\un{8\cdot 10^6}{MeV^3 fm^3})
=\un{0.71\cdot 10^{-4-6+45}}{MeV^{-3}} = \un{0.71\cdot 10^{35}}{MeV^{-3}}$
\item We want to multiply \un{600}{s} with factors of $\hbar$ and $c$ until we get inverse energy units. This is achieved by dividing by $\hbar$, which has units [energy]$\times$[time]. So we get $\un{600}{s} / \hbar$, which we calculate again with the $\hbar c$ trick, $\un{600}{s} /\hbar c/c = 600s /(\un{200}{MeV\,fm}) \cdot (3\cdot 10^8 m/s) = \un{9\cdot 10^{22}}{MeV^{-1}}$. 
\end{itemize}
So the solutions is:
"Take 2 eggs, a mass of \un{2.8\cdot{10^{35}}}{MeV} of flour and a volume of \un{0.71\cdot 10^{35}}{MeV^{-3}} of milk. Mix for \un{9\cdot 10^{22}}{MeV^{-1}}"
}}
}
\subsection{What's natural about that?*}
(Note the star - it means that this is material beyond the core programme of the course, but it's still useful for your understanding of the field and you're encouraged to read this. This particular section will hopefully get you to love natural units and see the deeper physics insights behind them.)

Especially those new to natural units often don't like them. Now, independent of
whether you happen to like them or not, you should probably get used to them
anyway, because if you want to communicate with other particle physicists,
you'll need to be able to use, or at least understand, natural units.  But I
think there's more to it.  Not only do natural units make your equations look
neater and therefore easier to follow, (keeping track of all those $\hbar$s
and $c$s can get quite tedious), but they are a consequence of real physical
insight. The reason we \emph{can} use the same units for space and time is
that space and time \emph{are}, to some degree, the same, and that is the
great insight of Special Relativity, and of Minkovsky space-time. It is not
just adding the $t$ to your usual 3-vector, a Lorentz boost really mixes those
time and space dimensions with each other.  And at the end it is
\[
c=\mathrm{constant},
\]
the experimental result that is at the very heart of Special Relativity, that
allows us to actually measure times in terms of the distance travelled by
light, or distances in units of time. The current definition of the good old
meter is based on how far light travels in a given amount of time. But if one
is defined in terms of the other anyway, why use two different units?

Apart from Relativity, the other big leap forward in the physics of the 20th
century was quantum mechanics, and one of its most important, and earlierst
equations is
\[
   p = \hbar \frac{2\pi}{\lambda}
\]
giving us a one-to-one relationship  between (inverse) distance and
momentum. Again it is the fundamental nature of the equation, and the
insight that $\hbar$ is a constant of nature, that allows us to measure
momenta in terms of inverse distances and vice versa - so why not give them
the same units? Similarly for energy and mass from
\[
 E = \gamma m c^2
\]
and energy and momentum from
\[
  E^2 = (cp)^2 + (mc^2)^2,
\]

Still not convinced? Take Newton's
\[
   ma = F = G \frac{Mm}{r^2}
\]
Note the repetition of the letter $m$. Each of the two instances
really means something quite different from the other: On the left
hand side it is the inertia of the body, i.e. its resistance to change
its velocity - the bigger $m$ is, the smaller is $a$ for the same
force $F$. We could call this $m_i$. On the right hand side, the $m$
stands for its power to attract another body, i.e. it is its
``gravitational charge" - it fulfils the same role as the electric
charge in Culoumb's law. We could call it $m_c$. It is
a great insight that for gravity, these two are proportional to each
other, $m_i \propto m_c$. And this is the reason we set the
proportionality constant between them to $1$, use the same units for
both, and even give them the same name - simply ``mass'', rather than
``inertial resistance'' and ``gravitational charge''. This seems
rather natural and the maths is far less clumsy - as with natural
units.
