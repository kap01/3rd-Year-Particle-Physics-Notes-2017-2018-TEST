


\subsection{Cabibbo Angle}
 It was found that the 
 theory of weak interactions explained well many features of quark and
 of leptonic decays. However, it appeared that the interaction
 strength in quark decays was less than one would expected, especially
 the strange quark seemed to be very reluctant to decay.

 To solve this problem, Cabibbo proposed that the down and the strange
 quark ``share'' the interaction strength/transition amplitude to the
 up quark \cite{cabibbo}. The idea is that there is a particle
 \emph{like} the down quark (let's call it $d^{\prime}$) that couples
 with the expected strength to the up quark, but is in fact a
 superposition of the physical $d$ and $s$ quark.
 \begin{equation}
     d^{\prime} = \alpha d + \beta s
 \end{equation}
 Normalising the $d^{\prime}$ wave function leads to the condition that $\alpha^2 + \beta^2=1$. This is fulfilled by:
 \begin{equation}
 d^{\prime} = \cos\theta_C \, d + \sin\theta_C s
\end{equation}
where $\theta_C$ is the Cabibbo angle (experimentally $\theta_C = 0.22$).
In this scheme, a $d^{\prime}, u, W$ vertex has the same coupling strength ($g_W$) as an $e, \nu_e, W$ vertex. 
But in reality, we do not observe $d^{\prime}$, but the physical $d, s$ quarks ("physics" meaning that they have well defined masses. 
The amplitude for a $d^{\prime}$ become a $d$ quark is $\braket{d^{\prime}, d} = \cos\theta_C$. Similarly $\braket{d^{\prime}, s} = \sin\theta_C$.
 So the transition amplitude for $d \to u$ transitions is $\propto
 \cos\theta_C$ and for $s\to u$ transitions it is $\propto \sin\theta_C$
 (Remember that rates are $\propto |\mathrm{amplitude}|^2$).
 
\begin{figure}
\caption{The Cabibbo Angle.\label{fig:cabRot}}
\begin{tabular}{cc}
\parbox{0.45\textwidth}{
The down-type quarks that couple via the $W^{\pm}$ to the up-type quarks are rotated relative to their mass
eigenstates.}
&
\parbox{0.45\textwidth}{
The $W^{\pm}$ can induce a transition $u\to d'$. The transition amplitudes to the mass eigenstates $d, s$  are $\propto$ the
projections of the mass eigenstates onto the $d'$.
}
\\
\includegraphics[width=0.45\textwidth]{fig/C_P_CP/cabibboAngle.png}
&
\includegraphics[width=0.45\textwidth]{fig/C_P_CP/cabbibo_with_su.png}
\end{tabular}
\end{figure}
 For symmetry reasons, let us also define an $s'$ such that $d', s'$ are simply rotates states of $d, s$ (this is somewhat anticipating the addition of the charm quark below - an $s'$ doesn't really make sense unless it is the partner of 2nd up-type quark):
\begin{equation}
\vII{d^{\prime}}{s^{\prime}} = \mII{\cos\theta_C}{\sin\theta_C}
                                  {-\sin\theta_C}{\cos\theta_C}
\vII{d}{s}
\end{equation}
as illustrated in \figref{fig:cabRot}.

The introduction of quark mixing explained the observed
 $s\leftrightarrow d$ and $u \leftrightarrow d$ couplings, and related them to $W^{\pm}$ couplings in the lepton sector.
 
 So the vertex factors are now:\\
 \includegraphics[width=0.5\textwidth]{fig/weak/gWCabibbo}
 
 So instead of three coupling constants, we have one coupling constant $g_W$ and one angle, $\theta_c \sim 13.02^{\circ}$ - that is one fewer parameter. Describing nature with fewer parameters is usually a sign of progress in our understanding of it.
 
 Note that, while $u \leftrightarrow s$ and $u \leftrightarrow
 d$ transitions are allowed, no vertices for $s \leftrightarrow d$ transitions are allowed (even though there is a neutral $Z$ boson that could take care of the charge conservation). This is explored a bit more deeply in the following optional section.
\subsection{No flavour changing neutral currents at tree level*}
The quark mixing that allows now the transition between up and down as well as up and strange does not introduce flavour changing couplings to the $Z$.

For this section we need some math that is beyond the scope of this course (you'll learn about this in the 4th year particle physics course). To follow the rest of this (starred - meaning non examinable) subsection, I need to ask you to accept the following without proof:
\begin{enumerate}[i)]
\item A function called "the Lagrangian" encodes all the physics there is. It is a function of fields. Each type of particle has an associated field. Particles are excitations of these fields.
\item All we need to know for this section is that, whenever there is a term with three particle fields in the Lagrangian, there is a corresponding Feynman rule for a 3-prong vertex (btw, terms with four particle fields give a 4-prong vertex etc). E.g. ABC leads to a vertex like this:
\\
\parbox[c]{0.3\textwidth}{\includegraphics[width=0.3\textwidth]{fig/C_P_CP/ABC_vtx}}
\\
(and also like this:
\parbox[c]{0.2\textwidth}{\includegraphics[width=0.2\textwidth]{fig/C_P_CP/CAB_vtx}}
and this:
\parbox[c]{0.2\textwidth}{\includegraphics[width=0.2\textwidth]{fig/C_P_CP/BCA_vtx}}.)
\end{enumerate}

 The terms describing the interaction of the down-type quarks with the $Z$ are proportional to $Z\bar{d}d + Z\bar{s}s$. Replacing now $d, s$ with $d', s'$ we get:
\begin{eqnarray}
 Z\bar{d}'d' + Z\bar{s}'s'
&=&
 Z\left(-\bar{s}\sin\theta_C + \cos\theta_C \bar{d}\right)
 \left(-s\sin\theta_C + \cos\theta_C d\right)
\nonumber\\
&&+
 Z\left( \bar{d}\sin\theta_C + \cos\theta_C \bar{s}\right)
 \left( d\sin\theta_C + \cos\theta_C s\right)
\nonumber\\
&=&
  Z\bar{s}s \left(\sin^2\theta_C + \bar{d}d \cos^2\theta_C
- (\bar{s}d + \bar{d}s) \sin\theta_C\cos\theta_C\right)
\nonumber\\
&=&
  Z\left(\bar{d}d \sin^2\theta_C + \bar{s}s \cos^2\theta_C
+ (\bar{s}d + \bar{d}s) \sin\theta_C\cos\theta_C\right)
\nonumber\\
&=& Z\bar{d}d + Z\bar{s}s 
\end{eqnarray}
So the basis change that we did for the down-type quark relative to the up quark does nothing to the $Z$ couplings, they are exactly as before (and forbid flavour changing interactions, i.e. no $d \to s$ transitions) at tree levels.

Note that the notion of an $s'$ quark, which is needed for the FCNC to cancel, only makes sense if there is also another up-type quark, the charm. So this is actually already an aspect of the GIM mechanism, described below.

\section{New Physics in 1963: Discovery of charm}
 (1963, quarks known at the time: $u, d, s$)\\ 
 In the previous chapters you learnt about the evidence for the existence of three quarks, which beautifully described the otherwise messy picture of the mesons and baryons observed. The key aspects of the theory of weak interactions (that you also learnt about last year), were also developed at a time when only three quarks were known: $u, d, s$. The 4th quark, charm, was predicted from observations made in decays of Kaons. Note that this is quite remarkable. Kaons have a mass of less than \un{0.5}{GeV}, while the charm quark has a mass of \un{1.3}{GeV} -- so how can you see a charm quark in a Kaon decay? With this section, we introduce an important general principle of an approach that is called "flavour physics": The discovery of "new physics" at mass scales beyond those of the particles that can be directly produced and observed, using precision measurements.


\subsection{Flavour changing neutral currents at loop level}
\begin{figure}
\centering
\includegraphics[width=0.9\textwidth]{fig/0_K2mumu1}
\caption{Diagrams contributing to the "Flavour Changing Neutral Current" (FCNC) transition \prt{K^0 \to \mu^+ \mu^-} (w/o GIM cancellation) leading to predicted rates far higher than observed. It is an FCNC because it is a $d-s$ annihilation, so a $d$ couples (indirectly) to an $s$ which has the same charge.
\label{fig:K2mumu1}}
\end{figure}
 The theory as it stands now has one substantial problem
 suffered from one problem, and that was the prediction of ``Flavour
 Changing Neutral Currents'', or FCNC, transitions where the quark
 flavour changes, but not the charge, like $d\to s$. Those were already impossible at tree level, but loop diagrams such as those in \figref{fig:K2mumu1} for the process \prt{K^0 \to \mu^+\mu^-} are still possible. This had never been seen and the predicted rate for this process was far greater than the observed limits at the time.
 

\subsubsection{GIM}
\begin{figure}
\centering
\includegraphics[width=0.9\textwidth]{fig/0_K2mumuGIM}
\caption{GIM mechanism illustrated for the "Flavour Changing Neutral Current" transition \prt{K^0 \to \mu^+\mu^-}. The diagrams resulting from the newly predicted $c$ quark exactly cancel the diagrams with the $u$ quark in the loop. Well - nearly exactly. The mass difference between the $c$ and the $u$ quark leads to slightly different values for the $u$ and the $c$ quark loop diagrams, so while \prt{K^0 \to \mu^+ \mu^-} is highly suppressed, it is not completely forbidden. This effect can even be used to estimate the $c$ quark mass. This mechanism works for any FCNC involving the first two generations (we'll deal with the third generation, soon).
\label{fig:K2mumuGIM}}
\end{figure}
 To explain the absence of flavour changing neutral currents
 S. L. Glashow,
 J. Iliopoulos and L. Maiani (``GIM'') proposed the existence of a
 fourth, undiscovered quark, the charm quark \cite{physrev:gim}. The
 charm quark would couple to the $s'$ quark via the $W$ in the same way as the $u$ quark does to the $d'$. This would have the effect of cancelling flavour changing neutral currents, basically due to the $-\sin\theta_C d$ contribution in the $s'$ that exactly cancels the $+\sin\theta_C s$ contribution of the $d'$ quark. This is illustrated for \prt{K^0 \to \mu^+\mu^-} in \figref{fig:K2mumuGIM}. 

The actual expression for the sum of the diagrams of the top row of \figref{fig:K2mumuGIM} and the corresponding diagram from the bottom row of  \figref{fig:K2mumuGIM}, is given by
\[
\mathcal{M}\sim cos\theta_{C}sin\theta_{C}m_{u}-sin\theta_{C}cos\theta_{C}m_{c}=cos\theta_{C}sin\theta_{C}(m_{u}-m_{c}).
\]
Therefore, if $m_{c}=m_{u}$ the overall diagram of $K^0\to\mu^+\mu^-$ decays is zero. However as a non-zero decay rate was measured, the measurement of the rate of $K^0\to\mu^+\mu^-$ processes can give an estimate of the mass of the charm quark.

With this addition, Cabibbo quark mixing and the GIM mechanism described the observed data well.

\fbox{\parbox{0.95\textwidth}{\textbf{Summary}:
With the $c$ quark have now two pairs (generations) of quarks:
\begin{equation}
\vII{u}{d^{\prime}},\;\;
\vII{c}{s^{\prime}}
\end{equation}
The $W^{\pm}$ takes a $u$ quark to a $d'$ quark and back, or a $c$ quark to a $s'$. The mass eigenstates are $u, c$ and $d, s$ (not $d', s'$), with $d = \cos\theta_C d' - \sin\theta_C s'$ and $s = \cos\theta_C s' + \sin\theta_C d'$, where $\theta_C$ is the famous Cabibbo angle, with $\sin\theta_C \approx 0.225$. The addition of the charm quark cancels the contribution of the $d$ quark in loop diagrams responsible for FCNC interactions. This cancellation would be exact if the $u$ and the $c$ quark were identical. They aren't, however. The only difference between $u$ and $c$ quarks is the mass of the $c$ and the diagram is roughly given by
\[
  cos\theta_{C}sin\theta_{C}(m_{u}-m_{c})
\]

Flavour changing neutral currents (FCNCs) are still highly suppressed, but can happen, with a very low probability that is related to the $u-c$ mass difference. This led to the prediction of the $c$ quark mass of about \un{1-3}{GeV}, quite remarkable given that no real ("on-shell") $c$ quark had been produced for this measurement.

Read in \secref{sec:CPV_CP} how the third generation was with top-and bottom quark, which are even heavier, was also predicted from flavour physics (even before the charm quark had been discovered). 
}}\\

\subsubsection{Charm Discovery}
 The subsequent discovery of the charm quark predicted by GIM was one
 of the great moments in particle physics. We however will just accept
 its existence and in fact will find evidence for further new quarks even before the charm quark was directly observed in the next section, when we connect
 quark mixing and \cp\ violation.  The discovery of
 the charm quark is described in Martin \& Shaw, ``Particle Physics'',
 pp 70 \cite{martin.shaw:pp}.
