\section{Introduction}
\subsection{Course content}
The goal of this course is to expand on ideas
introduced during your $2^{\rm nd}$ year Nuclear and Particle physics 
course regarding fundamental interactions in nature. Special focus will be given to how (non-)conservation of various quantum numbers gives rise to our understanding of the strong-, weak-nuclear and electromagnetic forces. Topics will include 
how our understanding of the strong force has evolved from the discovery of the pion and strong-isospin, to quarks, gluons and to modern particle physics. Will will also discuss how the measurements of quark transitions and Charge conjugation-Parity violation (CP violation), cemented our understanding of the weak force, and led to the prediction of new (and now familiar) particles. Finally will touch on the role of the Higgs boson and if time permits neutrino oscillations.

Key measurements which lead to breakthroughs in the field will be discussed, as well as key experimental techniques of particle detection.

\subsection{Course particulars}
In total the course consists of 18 lectures and 3 two-hour problems classes 
(you should turn up for these!!). Regarding reading material, the course should be self contained however you can find additional useful information in:
\begin{itemize}
\item Particle Physics (Martin and Shaw)
\item Introduction to elementary particles (Griffiths) 
\end{itemize}
Office hours: \\
Kostas: Fridays 13:30-14:30, office 4.54\\
Jonas: Wednedays 14:00-15:00, office 4.51\\
If you would like to meet outside these hours, that's fine, but please send an email to arrange a time.

\paragraph{Exercises}: Throughout these notes you will find a set of exercises which are meant both to familiarise yourselves which the material, as well as to get a chance to learn something interesting, beyond the scope of the course. Exercises which are meant to go above and beyond the syllabus will be clearly marked.

\paragraph{Exam}: January. You will have 2 hours  to complete two sections. Section-A: all parts are compulsory. Section-B: you get to choose one of two questions. There'll be a mock exam.

\paragraph{This document} contains a few sections of further reading beyond the scope of the course. These are marked with an asterisk in the section title. While they are not required reading, they still provide useful information that should aid your understanding of the subject.
%%
